\documentclass[12pt,A4]{article}
\ifx\pdftexversion\undefined
  \usepackage[dvips]{graphicx}
\else
  \usepackage[pdftex]{graphicx}
   \DeclareGraphicsRule{*}{mps}{*}{}
\fi

\usepackage[T1]{fontenc}
\usepackage{ae,aecompl}
\usepackage{times}

% Selects font encoding

\usepackage{url}
% breakes long urls prettier over more than one line

\usepackage[colorlinks=true,pageanchor=true,linkcolor=black,anchorcolor=black,filecolor=black,citecolor=black,menucolor=black,pagecolor=black,urlcolor=black,bookmarksopen=true,bookmarksopenlevel=1]{hyperref}
% Creates hyperlinks of references/urls. Default color is blue but the settings above change the colors to black.

\usepackage{parskip}
% Starts new paragraphs without indentation but with some space between the new and the previous paragraph.

\usepackage{multirow}
% Used to create tables with rows/cols spanning over se

\usepackage{graphicx}
% Used to include images with the includegraphics command

\usepackage{pdfpages}
% Used to import pages from or whole pdf-documents

\usepackage{listings}
\usepackage{listings}
\usepackage{color}
\usepackage{textcomp}
\definecolor{listinggray}{gray}{0.9}
\definecolor{lbcolor}{rgb}{0.9,0.9,0.9}
%http://www.tjansson.dk/?p=419
\lstset{
	backgroundcolor=\color{lbcolor},
	tabsize=4,
	rulecolor=,
	language=XML,
		basicstyle=\scriptsize,
        upquote=true,
        aboveskip={1.5\baselineskip},
        columns=fixed,
        showstringspaces=false,
        extendedchars=true,
        breaklines=true,
        prebreak = \raisebox{0ex}[0ex][0ex]{\ensuremath{\hookleftarrow}},
        frame=single,
        showtabs=false,
        showspaces=false,
        showstringspaces=false,
        identifierstyle=\ttfamily,
        keywordstyle=\color[rgb]{0,0,1},
        commentstyle=\color[rgb]{0.133,0.545,0.133},
        stringstyle=\color[rgb]{0.627,0.126,0.941},
        %otherkeywords={!,!=,~,$,*,\&,\%/\%,\%*\%,\%\%,<-,<<-,/}        
}

\usepackage{makeidx}
\makeindex
% Needed to create glossaries


\usepackage[number=none]{glossary}
\makeglossary
% To create glossary and list of acronyms. Other packages may also be used. Page numbering is turned off in the final list.

% Creates a new glossary for abbreviations
\newglossarytype[abr]{abbr}{abt}{abl}

\newglossarytype[alg]{acronyms}{acr}{acn}
\newcommand{\abbrname}{Abbreviations} 
\newcommand{\shortabbrname}{Abbreviations}
\newcommand{\enie}{ñ}
\makeabbr

\newcommand{\jpos}{jPOS }

\newcommand{\PM}{Presentation Manager }

% The below code sets the margins of the document.
% See: http://web.image.ufl.edu/help/latex/margins.shtml for an explanation of the margins.

\oddsidemargin 5mm
\evensidemargin 5mm
\textwidth 150mm
\topmargin 0mm
\headheight 0mm
\textheight 225mm
%\footheight 0mm

\usepackage{color}
% To give color to the text. It is only used to highlight the instructions in this document and it can be removed.

\usepackage{float}
\newcommand{\myfigure}[4]{
	\begin{figure}[H]
	\begin{center}
	\includegraphics[width=#1\textwidth]{img/#2} 
	\caption[#3]{#3}
	\label{#4} 
	\end{center}
	\end{figure}
}

\begin{document}

\begin{titlepage}
\begin{center}

\vspace*{2cm}

\huge{jPOS Presentation Manager}

\vspace{.5cm}
\large{Jeronimo Paoletti}

\vspace{1cm}

\vspace{1cm}
\huge{User Manual}


\vspace{6cm}

\end{center}
\normalsize

\begin{table}[!h]
\begin{tabular}{ll}
\multirow{4}{*}{\includegraphics[width=20mm]{img/jpos.jpg}} & \\
&  \\
& La Plata, Buenos Aires, Argentina \\
& April 2010 \\

\end{tabular}
\end{table}

\end{titlepage}

\clearpage

\pagestyle{plain}

\tableofcontents

\clearpage

\section{Introduction}


\subsection{Motivation}

If you already used jPOS, for sure you were shocked (like me) because of its
great arquitecture and the number of tools to make electronic transaction
systems, but probably, like me, you have a business model with a lot of
participants on it associated to your transacitional model and this information
must be administrated so you will need to probably start an extension of the
transactional system to give it an interface. This will make you (or your
transactional experts developers) loose time doing some Web interface or
something to administrate business objects. More over, probably you will start
with transactions and sadly, this is something ``ethereal'' to most of the
clients. This is a problem and this project is intended to help you and any
jPOS developer to give this a good ``jPOS style'' solution.

\subsection{What is jPOS-Presentation Manager}

Its a group of jPOS modules based on an xml representation of user model
objects (entities). The goal of this is to provide a generic, extensible, easy
to start and to expand, way of generate an administrative interface for jPOS
based transactional projects.

Trying to give a solution to the problem presented before, jPOS-PM gives a lot
of tools to give a jPOS system a fast Web based start. With a few configuration
you can have a full web site with a simple yet extensible interface so you can
concentrate your work on transaction, make a little and fast start to the
interface and get a nice start with few extra work. When transactional part is
ready, you can make a lot of enhancement on Web interface with the many
features jPOS-PM have.

\subsection{How can I get it}

You can get it from GitHub project.

\url{http://github.com/jpaoletti/jPOS-Presentation-Manager}

\subsection{What it needs?}

jPOS-PM requires this jPOS modules to work:

\begin{description}
\item[jpos] \url{http://jposee.googlecode.com/svn/trunk/modules/jpos/}
\item[constants] \url{http://jposee.googlecode.com/svn/trunk/opt/}
\item[jetty6] If you are going to work with struts implementation (default),
you need \url{http://jposee.googlecode.com/svn/trunk/opt/}
\item[commons] \url{http://jposee.googlecode.com/svn/trunk/opt/}
\end{description}

\section{Entity Management}

\subsection{Module structure}

\subsection{Configuring Bean}

\begin{lstlisting}[language=XML]
<presentation-manager class="org.jpos.ee.pm.struts.PMStrutsService"  logger="Q2">
    <property name="debug" value="true" /> 
    <property name="template" value="default" />
    <property name="appversion" value="0.9.6.0" />
    <property name="login-required" value="true" />
    <property name="ignore-db" value="false" />
    <property name="title" value="project.title" />
    <property name="subtitle" value="project.subtitle" />
    <property name="default-data-access" value="org.jpos.ee.pm.core.DataAccessDB"/>
    
    <!-- Busissnes Entities -->
    <property name="entity" value="cfg/entities/sysconfig.xml" />
    <!--     Security Entities -->
    <property name="entity" value="cfg/entities/secuser.xml" />
    <property name="entity" value="cfg/entities/secusergroup.xml" />
    <property name="entity" value="cfg/entities/secperm.xml" />
    <property name="entity" value="cfg/entities/secuserprofile.xml" />

    <property name="monitor" value="cfg/monitors/q2.xml" />
    <property name="monitor" value="cfg/monitors/status.xml" />

</presentation-manager>
\end{lstlisting}

\subsection{Defining Entities}

Example: SysConfig

\begin{lstlisting}[language=XML]
<?xml version='1.0' ?>
<!DOCTYPE entity SYSTEM "cfg/entity.dtd">  
<entity id="sysconfig" clazz="org.jpos.ee.SysConfig">
      <auditable>false</auditable>
      <operations>
          <operation id="list" enabled="true" scope="item" display="all">
            <properties>
                <property name="operation-column-width" value="1px" />
            </properties>
          </operation>
          <operation id="add" enabled="true" scope="general" display="all"/>
          <operation id="show" enabled="true" scope="item" display="all"/>
          <operation id="edit" enabled="false" scope="item" display="all"/>
      </operations>

      <field id="id" display="list add show" align="center" width="30" >
          <converters>
              <converter class="org.jpos.ee.pm.struts.converter.EditStringConverter" operations="add edit" />
          </converters>
      </field>
      <field id="value" display="all">
          <converters>
              <converter class="org.jpos.ee.pm.struts.converter.EditStringConverter" operations="add edit" />
          </converters>
      </field>
      <field id="readPerm" display="all" width="50">
          <converters>
              <converter class="org.jpos.ee.pm.struts.converter.EditStringConverter" operations="add edit" />
          </converters>
      </field>
      <field id="writePerm" display="all" width="50">
          <converters>
              <converter class="org.jpos.ee.pm.struts.converter.EditStringConverter" operations="add edit" />
          </converters>
      </field>
</entity>
\end{lstlisting}


pm.entity.sysconfig=System Configuration
pm.field.sysconfig.id=Id
pm.field.sysconfig.value=Value 
pm.field.sysconfig.readPerm=Read Permission
pm.field.sysconfig.writePerm= Write Permission



\subsubsection{Basic Definition}

\begin{lstlisting}[language=XML]

\end{lstlisting}

\subsubsection{Context}
\begin{lstlisting}[language=XML]

\end{lstlisting}
\subsubsection{Data Access}
\begin{lstlisting}[language=XML]

\end{lstlisting}
\subsubsection{Fields}
\begin{lstlisting}[language=XML]

\end{lstlisting}
\subsubsection{Operations}

List

\begin{lstlisting}[language=XML]
          <operation id="list" enabled="true" scope="item" display="all">
            <properties>
                <property name="operation-column-width" value="80px" />
                <property name="sort-field" value="id"/>
                <property name="sort-direction" value="desc"/>
                <property name="searchable" value="true"/>
                <property name="show-row-number" value="true" />
            </properties>
          </operation>
\end{lstlisting}



\subsubsection{Converters}
\begin{lstlisting}[language=XML]

\end{lstlisting}
\subsubsection{Validators}
\begin{lstlisting}[language=XML]

\end{lstlisting}



\section{Menu}

\emph{cfg/pm.menu.xml}
\begin{lstlisting}[language=XML]
<?xml version="1.0" encoding="UTF-8"?>
<!DOCTYPE menu SYSTEM "menu.dtd">
<menu>
    <menu-list text="pm.menu.list.monitor" perm="monitoring">
        <menu-item text="pm.menu.item.txns" perm="">
            <location id="pmstruts" value="/list.do?pmid=transaction" />
            <external>false</external>
        </menu-item>
        <menu-item text="pm.menu.item.q2log" perm="">
            <location id="pmstruts" value="/monitor.do?pmid=q2" />
            <external>false</external>
        </menu-item> 
        <menu-item text="pm.menu.item.status" perm="">
            <location id="pmstruts" value="/monitor.do?pmid=status" />
            <external>false</external>
        </menu-item> 
        <menu-item text="pm.menu.item.events" perm="">
            <location id="pmstruts" value="/list.do?pmid=event" />
            <external>false</external>
        </menu-item>
    </menu-list>
    <menu-list text="pm.menu.list.business" perm="business">
        ...
    </menu-list>
    <menu-list text="pm.menu.list.security" perm="useradmin">
        ...
    </menu-list>
    <menu-list text="pm.menu.list.config" perm="configuration">
        <menu-item text="pm.menu.item.sysconfig" perm="">
            <location id="pmstruts" value="/list.do?pmid=sysconfig" />
            <external>false</external>
        </menu-item>
        ...
    </menu-list>
    </menu-list>
</menu>
\end{lstlisting}

Foto con la key

pm.menu.item.sysconfig=System Configuration

foto con el correcto.

\subsection{Defining Locations}

\emph{cfg/pm.locations.xml}
\begin{lstlisting}[language=XML]
<?xml version="1.0" encoding="UTF-8"?>
<!DOCTYPE menu SYSTEM "locations.dtd">
<locations>
       <location id="pmstruts" class="org.jpos.ee.pm.struts.MenuItemLocationStruts"/>
</locations>
\end{lstlisting}

\section{Monitors}

\subsection{Defining Monitors}

\subsection{Watching Monitors}

\subsection{Extending Monitors}

\end{document}
