\documentclass[12pt,A4]{article}
\ifx\pdftexversion\undefined
  \usepackage[dvips]{graphicx}
\else
  \usepackage[pdftex]{graphicx}
   \DeclareGraphicsRule{*}{mps}{*}{}
\fi

\usepackage[T1]{fontenc}
\usepackage{ae,aecompl}
\usepackage{times}

% Selects font encoding

\usepackage{url}
% breakes long urls prettier over more than one line

\usepackage[colorlinks=true,pageanchor=true,linkcolor=black,anchorcolor=black,filecolor=black,citecolor=black,menucolor=black,pagecolor=black,urlcolor=black,bookmarksopen=true,bookmarksopenlevel=1]{hyperref}
% Creates hyperlinks of references/urls. Default color is blue but the settings above change the colors to black.

\usepackage{parskip}
% Starts new paragraphs without indentation but with some space between the new and the previous paragraph.

\usepackage{multirow}
% Used to create tables with rows/cols spanning over se

\usepackage{graphicx}
% Used to include images with the includegraphics command

\usepackage{pdfpages}
% Used to import pages from or whole pdf-documents

\usepackage{listings}
\usepackage{listings}
\usepackage{color}
\usepackage{textcomp}
\definecolor{listinggray}{gray}{0.9}
\definecolor{lbcolor}{rgb}{0.9,0.9,0.9}
%http://www.tjansson.dk/?p=419
\lstset{
	backgroundcolor=\color{lbcolor},
	tabsize=4,
	rulecolor=,
	language=XML,
		basicstyle=\scriptsize,
        upquote=true,
        aboveskip={1.5\baselineskip},
        columns=fixed,
        showstringspaces=false,
        extendedchars=true,
        breaklines=true,
        prebreak = \raisebox{0ex}[0ex][0ex]{\ensuremath{\hookleftarrow}},
        frame=single,
        showtabs=false,
        showspaces=false,
        showstringspaces=false,
        identifierstyle=\ttfamily,
        keywordstyle=\color[rgb]{0,0,1},
        commentstyle=\color[rgb]{0.133,0.545,0.133},
        stringstyle=\color[rgb]{0.627,0.126,0.941},
        %otherkeywords={!,!=,~,$,*,\&,\%/\%,\%*\%,\%\%,<-,<<-,/}        
}

\usepackage{makeidx}
\makeindex
% Needed to create glossaries


\usepackage[number=none]{glossary}
\makeglossary
% To create glossary and list of acronyms. Other packages may also be used. Page numbering is turned off in the final list.

% Creates a new glossary for abbreviations
\newglossarytype[abr]{abbr}{abt}{abl}

\newglossarytype[alg]{acronyms}{acr}{acn}
\newcommand{\abbrname}{Abbreviations} 
\newcommand{\shortabbrname}{Abbreviations}
\newcommand{\enie}{ñ}
\makeabbr

\newcommand{\jpos}{jPOS }

\newcommand{\PM}{Presentation Manager }

% The below code sets the margins of the document.
% See: http://web.image.ufl.edu/help/latex/margins.shtml for an explanation of the margins.

\oddsidemargin 5mm
\evensidemargin 5mm
\textwidth 150mm
\topmargin 0mm
\headheight 0mm
\textheight 225mm
%\footheight 0mm

\usepackage{color}
% To give color to the text. It is only used to highlight the instructions in this document and it can be removed.

\usepackage{float}
\newcommand{\myfigure}[4]{
	\begin{figure}[H]
	\begin{center}
	\includegraphics[width=#1\textwidth]{img/#2} 
	\caption[#3]{#3}
	\label{#4} 
	\end{center}
	\end{figure}
}